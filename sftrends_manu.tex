\documentclass[letterpaper,12pt,oneside]{article}\usepackage[]{graphicx}\usepackage[]{color}
%% maxwidth is the original width if it is less than linewidth
%% otherwise use linewidth (to make sure the graphics do not exceed the margin)
\makeatletter
\def\maxwidth{ %
  \ifdim\Gin@nat@width>\linewidth
    \linewidth
  \else
    \Gin@nat@width
  \fi
}
\makeatother

\definecolor{fgcolor}{rgb}{0.345, 0.345, 0.345}
\newcommand{\hlnum}[1]{\textcolor[rgb]{0.686,0.059,0.569}{#1}}%
\newcommand{\hlstr}[1]{\textcolor[rgb]{0.192,0.494,0.8}{#1}}%
\newcommand{\hlcom}[1]{\textcolor[rgb]{0.678,0.584,0.686}{\textit{#1}}}%
\newcommand{\hlopt}[1]{\textcolor[rgb]{0,0,0}{#1}}%
\newcommand{\hlstd}[1]{\textcolor[rgb]{0.345,0.345,0.345}{#1}}%
\newcommand{\hlkwa}[1]{\textcolor[rgb]{0.161,0.373,0.58}{\textbf{#1}}}%
\newcommand{\hlkwb}[1]{\textcolor[rgb]{0.69,0.353,0.396}{#1}}%
\newcommand{\hlkwc}[1]{\textcolor[rgb]{0.333,0.667,0.333}{#1}}%
\newcommand{\hlkwd}[1]{\textcolor[rgb]{0.737,0.353,0.396}{\textbf{#1}}}%
\let\hlipl\hlkwb

\usepackage{framed}
\makeatletter
\newenvironment{kframe}{%
 \def\at@end@of@kframe{}%
 \ifinner\ifhmode%
  \def\at@end@of@kframe{\end{minipage}}%
  \begin{minipage}{\columnwidth}%
 \fi\fi%
 \def\FrameCommand##1{\hskip\@totalleftmargin \hskip-\fboxsep
 \colorbox{shadecolor}{##1}\hskip-\fboxsep
     % There is no \\@totalrightmargin, so:
     \hskip-\linewidth \hskip-\@totalleftmargin \hskip\columnwidth}%
 \MakeFramed {\advance\hsize-\width
   \@totalleftmargin\z@ \linewidth\hsize
   \@setminipage}}%
 {\par\unskip\endMakeFramed%
 \at@end@of@kframe}
\makeatother

\definecolor{shadecolor}{rgb}{.97, .97, .97}
\definecolor{messagecolor}{rgb}{0, 0, 0}
\definecolor{warningcolor}{rgb}{1, 0, 1}
\definecolor{errorcolor}{rgb}{1, 0, 0}
\newenvironment{knitrout}{}{} % an empty environment to be redefined in TeX

\usepackage{alltt}
\usepackage[paperwidth=8.5in,paperheight=11in,top=1in,bottom=1in,left=1in,right=1in]{geometry}
\usepackage{setspace}
\usepackage[colorlinks=true,allcolors=Blue]{hyperref}
\usepackage[usenames,dvipsnames]{xcolor}
\usepackage{indentfirst}
\usepackage{titlesec}
\usepackage{multirow}
\usepackage{booktabs}
\usepackage{graphicx}
\usepackage{verbatim}
\usepackage{rotating}
\usepackage{tabularx}
\usepackage{outlines}
\usepackage{lineno}
\usepackage{array}
\usepackage{times}
\usepackage{cleveref}
\usepackage{acronym}
\usepackage[position=t]{subfig}
\usepackage{paralist}
\usepackage[noae]{Sweave}
\usepackage{natbib}
\usepackage{array}
\usepackage{pdflscape}
\usepackage{bm}
% \usepackage{showlabels}
\bibpunct{(}{)}{,}{a}{}{,}

% page margins and section title formatting
\linespread{1.5}
\setlength{\footskip}{0.5in}
\titleformat*{\section}{\Large\bf\em}
\titleformat*{\subsection}{\singlespace\large\bf}
\titleformat*{\subsubsection}{\singlespace\normalsize\bf\em}
\titlespacing{\section}{0in}{0in}{0in}
\titlespacing{\subsection}{0in}{0in}{0in}
\titlespacing{\subsubsection}{0in}{0in}{0in}

% cleveref options
\crefname{table}{Table}{Tables}
\crefname{figure}{Fig.}{Figs.}
\renewcommand{\figurename}{Fig.}

% aliased citations
\defcitealias{BeckIP}{Beck and Murphy in press}
\defcitealias{SutulaIR}{Sutula et al. in review}

%acronyms
\acrodef{chla}[chl-\textit{a}]{chlorophyll \textit{a}}
\acrodef{din}[DIN]{dissolved inorganic nitrogen}
\acrodef{emp}[EMP]{Environmental Monitoring Program}
\acrodef{iep}[IEP]{Interagency Ecological Program}
\acrodef{jas}[JAS]{July-August-September}
\acrodef{sfe}[SFE]{San Francisco Estuary}
\acrodef{wrtds}[WRTDS]{Weighted Regressions on Time, Discharge, and Season}
\acrodef{wwtp}[WWTP]{wastewater treatment plant}

%for supplemental figures/tables
\newcommand{\beginsupplement}{%
        \setcounter{table}{0}
        \renewcommand{\thetable}{S\arabic{table}}%
        \setcounter{figure}{0}
        \renewcommand{\thefigure}{S\arabic{figure}}%
     }

%knitr options


% get the version based on commit date


% get online bib file


\IfFileExists{upquote.sty}{\usepackage{upquote}}{}
\begin{document}

\raggedbottom
\linenumbers
\raggedright
\urlstyle{same}
\setlength{\parindent}{0.5in}
\renewcommand\refname{References \vspace{12pt}}

\begin{singlespace}
\title{{\bf {\Large Four decades of water quality change in the upper San Francisco Estuary}}}
\author{
  {\bf {\normalsize Marcus W. Beck$^1$, David Senn$^2$, Phil Bresnahan$^2$, Emily Novick$^2$, James D. Hagy III$^1$,}}
  \\{\bf {\normalsize Thomas Jabusch$^2$, Phil Trowbridge$^2$}}
  \\\\{\textit {\normalsize $^1$USEPA National Health and Environmental Effects Research Laboratory}}
  \\{\textit {\normalsize Gulf Ecology Division, 1 Sabine Island Drive, Gulf Breeze, FL 32561}}
	\\{\textit {\normalsize Phone: 850-934-2480, Fax: 850-934-2401}}
	\\{\textit {\normalsize Emails: \href{mailto:beck.marcus@epa.gov}{beck.marcus@epa.gov}, \href{mailto:hagy.jim@epa.gov}{hagy.jim@epa.gov}}}
  \\\\{\textit {\normalsize $^2$San Francisco Estuary Institute}}
	\\{\textit {\normalsize 4911 Central Avenue, Richmond, CA 94804}}
	\\{\textit {\normalsize Phone: 510-746-7334, Fax: 510-746-7300}}
	\\{\textit {\normalsize Emails: \href{mailto:davids@sfei.org}{davids@sfei.org}, \href{mailto:philb@sfei.org}{philb@sfei.org}, \href{mailto:emilyn@sfei.org}{emilyn@sfei.org}, \href{mailto:thomasj@sfei.org}{thomasj@sfei.org}}}
  \vspace{1in} 
  \\ Version Date:   Thu Nov 17 13:44:09 2016 -0600
	}
\date{}
\maketitle
\end{singlespace}
\clearpage

\begin{abstract}
\noindent Recent methods for trend analysis have been developed that leverage the descriptive potential of long-term time series.  Combined with these methods, multi-decadal datasets of water quality in coastal systems can provide valuable opportunities to gain insights into ecosystem properties and drivers of change.  This study describes use of an estuarine adaptation of the \ac{wrtds} model to describe water quality trends over four decades in the Delta region of the \ac{sfe}. This region is a complex mosaic of inflows that are primary sources of nutrients into the larger Bay.  To date, a comprehensive evaluation of flow-normalized trends using the long-term monitoring dataset at multiple stations in the Delta has not been conducted despite the importance of nutrient transport from the region for water quality in the entire bay.  The \ac{wrtds} technique is data-driven where the parameterization of the functional model changes smoothly over time following dynamic patterns of season and flow.  Water quality trends that have not been previously quantified can be described, including variation in flow-normalized concentrations, frequency occurrence of extreme events, and response to historical changes in the watershed, all of which are important needs for understanding changes in the \ac{sfe}.  Model results from multiple stations in the Delta provided novel descriptions of historical trends and relationships between key species of dissolved inorganic nitrogen (ammonium, nitrate/nitrite, total).  This variation was described in the context of varying contributions of input flows from the Sacramento and San Joaquin rivers, as well as tidal exchange with the central \ac{sfe}.  Conceptual relationships between water quality and drivers of change were used to generate and test hypotheses of mechanistic relationships using selected examples from the trend descriptions. Overall, this analysis provides an ecological and management-based understanding of historical trends in the \ac{sfe} as a means to interpret potential impacts of recent changes and expected trends in this dynamic system.  An argument is also made for more comprehensive evaluations of long-term monitoring datasets to understand relationships between response endpoints and causal mechanisms in coastal waters.
\end{abstract}
\acresetall

\section{Introduction}

\begin{enumerate}
\item How and why are trends interpreted - assessment of raw data, surrogates, various methods (kendall, GAM, WRTDS), what have been implications of using different approaches, see Kratzer USGS report http://pubs.usgs.gov/sir/2010/5228/pdf/sir20105228.pdf and data http://pubs.usgs.gov/sir/2010/5228/, need to interpret eutrophication trends in estuaries - it's confusing \citep{Cloern10}
\item WRTDS, original method \citep{Hirsch10,Hirsch15}
\item WRTDS application to Tampa Bay as test set \citep{Beck15}, further validation in Patuxent and other tidal waters \cite{Beck15b}
\item SF estuary, unique and prominent location, full story is complex (historical context and recent changes) \citep{Cloern12b}, why is the delta important (a vigorous biogeochemical reactor) \citep{Jassby00,Jassby02,Jassby08}, no one has empirically described the data in the delta using data-intensive methods

San Francisco Bay on the Pacific Coast of the United States is one of the most prominent estuaries in the western hemisphere.  Background nutrient concentrations in the Bay often exceed those associated with excessive primary production, although eutrophication events have historically been infrequent.  Recent changes in response to additional stressors (e.g., variation in freshwater inputs/withdrawals, invasive species, climate change) suggests that Bay condition has not followed historical trajectories and more subtle spatial and temporal variation could provide clues that describe underlying properties of this system.  The unique ecological and social context of the Bay, including a rich source of monitoring data from the last four decades, provides a valuable opportunity to gain insight into ecosystem properties of estuaries. 

\item Study goal and objectives
\begin{itemize}
\item Provide a description of trends - annual, seasonal, spatial, response to flow, change by analytes
\item Detailed description of selected sites in the context of conceptual relationships - 1) nonlinear or extreme quantile changes, site TBD, 2) P8 and WWTP improvements, 3) Suisun DIN, SiO2, Chla, and clams
\item What this means for understanding other systems
\end{itemize}
\end{enumerate}

\section{Methods}

\subsection{Study system}

The \ac{sfe} drains a 200 thousand km$^2$ watershed and is the largest bay on the Pacific coast of North America.  The watershed provides drinking water to over 25 million people, including irrigation for 18 thousand km$^2$ of agricultural land in the Central Valley.  Water enters the Bay through the Sacramento and San Joaquin rivers that have a combined inflow of approximately 28 km$^3$ per year, with the Sacramento accounting for 84\% of inflow to the Delta.  The \ac{sfe} system is divided into several sub-bays, including Suisun Bay immediately downstream of the Delta, San Pablo Bay to the north, South Bay, and the Central Bay that drains to the Pacific Ocean through the Golden Gate.  Water dynamics in \ac{sfe} are governed by inflows from the watershed, tidal exchange with the Pacific Ocean, and water withdrawals for municipal and agricultural use \citep{Jassby00}.  Seasonally, inflows into \ac{sfe} peak in the spring and early summer from snowmelt in the upper watershed, whereas consumption, withdrawals, and export have steadily increased from 1960 to present but vary considerably depending on inter-annual climate effects \citep{Cloern12b}.  The system is mixed mesotidal and significant exchange with the ocean occurs daily, although the extent of landward saltwater intrusion varies with inflow and annual water use patterns. Notable drought periods have occurred from 1976-1977, 1987-1992, and recently from 2013-2015 \citep{Cloern15}.  Oceanic upwelling and climatic variation are also significant external factors that have influenced water quality dynamics in the Bay \citep{Cloern07}.

Nutrient loading in \ac{sfe} is comparable to other large estuaries that exhibit symptomatic effects of cultural eutrophication \citep[e.g., Chesapeake Bay,][]{Kemp05}.  Orthophosphate (PO$_4^{3-}$) and \ac{din} enter the Bay primarily through riverine sources in the north and municipal \ac{wwtp} inputs in the densely-populated area immediately surrounding \ac{sfe}.  Annual nutrient export from the Delta region has been estimated as approximateily 30 thousand kg d$^{-1}$ of total nitrogen \citep[varying with flow,][]{Novick15}, with 90\% of ammonium (NH$_4^{+}$) originating solely from the Sacramento Regional \ac{wwtp} \citep{Jassby08}.  Although nitrogen and phosphorus inputs are considerable, primary production is relatively low and not nutrient-limited \citep{Jassby02,Kimmerer12}.  The resistance of \ac{sfe} to the negative effects of eutrophication has historically been attributed to the unique physical and biological characteristics of the Bay, including strong tidal mixing that limits stratification \citep{Cloern96,Thompson08} and limits on phytoplankton growth from high turbidity and filter-feeding by bivalve mollusks \citep{Thompson08,Crauder16}.  However, recent water quality trends have suggested that resistence of the system to nutrient inputs is decreasing given documented increases in chlorophyll biomass \citep{Cloern07}, increased occurrence of hypoxic conditions \citepalias{SutulaIR}, and increased abundance of phytoplankton species associated with harmful algal blooms \citep{Lehman05,Lehman10}.  These recent changes have been attributed to variation in global sea surface temperatures associated with climate change \citep{Cloern07}, biological invasions \citep{Cohen98}, and departures from the historical flow record \citep{Enright09,Cloern12}.  

The Delta region is of particular interest for understanding historical patterns and potential trajectories of water quality response to nutrient inputs into the Bay.  The Delta is a mosaic of linked channels or tracts that receive, process, and transport inflows from the Sacramento and San Joaquin rivers \citep{Jassby00,Jassby08,Novick15}.  Quantitative descriptions of nutrient dynamics in the Delta are challenging given the numerous sources of nutrients and the volume of water that is exchanged through natural and anthropogenic processes.  A comprehensive evaluation using mass-balance models to describe nutrient dynamics in the Delta demonstrated that nitrogen enters the system in different forms and is processed at different rates before export or removal \citep{Novick15}. For example, a majority of ammonium entering the system during the summer is nitrified or assimilated, whereas a considerable percentage of total nitrogen load to the Delta is lost.  Although, the focus of our analysis is not to quantify sources or sinks of nitrogen species, a quantitative evaluation of long-term trends will provide a more comprehensive historical interpretation to hypothesize the effects of future changes in the context of known dynamics.  Nutrients in the Delta also vary with seasonal and annual changes in the delivery of water inflows, including water exports directly from the system \citep{Jassby00,Jassby08}.  Our analysis also explicitly accounts for the effects of flow changes on nutrient response to better understand variation both within the Delta and potential mechanisms of downstream tranport. 

\subsection{Data sources}

Multi-decadal time series of nutrients and flow records were used to develop a quantitative description of nitrogen trends in the Delta.  The \ac{iep} is a consortium of state and federal agencies that have maintained the \ac{emp} in the Delta region since 1975 \citep{IEP13}.  The \ac{emp} collects monthly water quality samples at 19 stations in the Delta, Suisun Bay, and northeastern San Pablo Bay.  Water samples were collected using a Van Dorn sample, a submersible pump, or a flow through system depending on site.  All samples were processed with standard QA/QC at the California Department of Water Resources Bryte Laboratory in Sacramento \citep[references in][]{IEP13}.  Nutrient time series were obtained from the \ac{iep} website (\url{http://water.ca.gov/bdma/meta/Discrete/data.cfm}) at nine discrete sampling stations from 1976 to 2012.  Three representative stations from  three locations in the Delta and Suisun Bay were used: Delta stations C3 (Sacramento inflow), C10 (San Joaquin inflow), P8; middle stations D19, D26, D28; and Suisun stations D4, D6, and D7.  These stations were chosen based on continuity of the water quality time series and geographic location for understanding trends.  Data were minimally processed with the exception of averaging replicates that occurred on the same day.  Detection limits throughout the period of record were obtained from the metadata, although few observations were censored ($<$ 3\%).  The three nitrogen analytes that were evaluated were ammonium, nitrite/nitrate, and \ac{din} (as the sum of the former two).  

Daily flow estimates for the Delta region were obtained from the Dayflow software program that provides estimates of average Delta outflow \citep{IEP16}.  Because of the complexity of water inflow, exports, and outflows from the Delta, the Dayflow program combines observations with estimates based on mass balance to reconstruct historical and daily flow estimates.  The \ac{wrtds} models described below require a matched flow record with the appropriate station to evaluate nutrient trends. Given the complexity of inflows and connectivity of the system, only the inflow estimates from the Sacramento and San Joaquin rivers were used as measures of freshwater influence at each station.  Initial analyses indicated that model fit was not signifcantly improved with flow estimates from locations closer to each station, nor was model fit improved using lagged times series.  As such, the Sacramento daily flow time series was used to acount for flow effects at C3, D19, D26, and D28, and the San Joaquin time series was used for C10 and P8.  The salinity observations at D4, D6, and D7 in Suisun Bay were used as a more appropriate measure of variation in freshwater balance given the stronger tidal influence at these stations.  Salinity has been used a tracer of freshwater influence for the application of WRTDS models in tidal waters \citep{Beck15}.  

\subsection{Analysis method and application}

A total of twenty-seven \ac{wrtds} models were created, one for each nitrogen at each station.  The functional form of \ac{wrtds} is a simple regression that models the log-transformed response variable as a function of time, flow, and season:

\begin{equation}
\ln\left(N\right) = \beta_0 + \beta_1 t + \beta_2 \ln\left(Q\right) + \beta_3 \sin\left(2\pi t\right) + \beta_4 \cos\left(2\pi t\right)
\end{equation}  

\noindent where $N$ is one of three nitrogen analytes, time $t$ is a continuous variable as decimal time to capture the annual or seasonal trend, and $Q$ is the flow variable (either flow or salinity depending on station).  The seasonal trend is modelled as a sinusoidal component to capture periodicity between years.  The \ac{wrtds} model is a moving window regression that fits a unique set of parameters at each observation point in the time series.  A unique set of weights is used for each regression to control the relevance of observations used to fit the model to the observation at the center of the window. The weights are based on a scaled Euclidean distance to estimate the differences of all points from the center in relation to annual time, season, and flow.  The final vector used to fit the model at each point weights observations more similar to the center of the window with more importance.  The complete model for the time series contains a parameter set for every time step that considers the unique context of the data.  As such, predictions from \ac{wrtds} are more precise than those from more conventional models that fit a single parameter set to the entire time series.  The original \ac{wrtds} method is described in more detail in \citep{Hirsch10}.  The \ac{wrtds} model applied to the Delta time series followed methods in \citet{Beck15}, which were based on a tidal adaptation of the original method.  The \ac{wrtds} models were fit to describe the conditional mean response using a weighted Tobit model for left-censored data \citep{Tobin58}.  Previous adaptations of \ac{wrtds} to tidal waters have used quantile regression to describe trends in the conditional quantiles, such as changes in the frequency of occurrence of extreme evens.  The application to the Delta data focused only on the conditional mean models to establish a baseline response which has not been previously quantified.  All analyses used the WRTDStidal package for R \citep{Beck16b,RDCT16}

A hallmark of the \ac{wrtds} approach is the description of flow-normalized trends that are independent of variation from freshwater inflows.  Flow-normalized trends have value for the interpretation of changes that are potentially caused by drivers other than  flow, such as \ac{wwtp} upgrades or phytoplankton grazing by benthic filter-feeders (\cref{fig:schematic}, \citealt{Beck15}). Although variation in nutrients is caused by the combined effects of several variables acting at different temporal and spatial scales, flow-normalization provides a basis for further exploration by removing a critical confounding variable that could affect the interpretation of trends. A flow-normalized value is the average of predictions at a given observation using all flow values that are expected to occur for the relevant month across years in the record.  Flow-normalized trends for each analyte at each station were used to describe long-term changes in different annual and seasonal periods.  Specifically, flow-normalized trends in each analyte were summarized as both averages and percent changes from the beginning to end of annual groupings from 1976-1995, 1996-2000, 2001-2008, and 2009-2014, and seasonal groupings of March-April-May (spring), June-July-August (summer), September-October-November (fall), and December-January-February (winter). These annual and seasonal groupings were chosen for continuity with additional research efforts \citep[i.e.,][]{Jabusch16}.  Trends within each seasonal grouping were based on annual mean estimates for the first and last three years of flow-normalized estimates to reduce the effects of abnormal years or missing data.  Annual trends were based only on the first and last years in the time period due to limited number of years in some groupings \citepalias{BeckIP}.

\subsection{Case studies}

Three stations were chosen for closer evaluation to demonstrate use of \ac{wrtds} to develop a more detailed understanding of trends for further evaluation.  The stations werechosen to address ecological and management-based questions that have relevance outside of the region, having importance for our understanding of estuarine processes that influence eutrophication trends on decadal time scales.  The selected case studies focused on \begin{inparaenum}[1\upshape)]
\item nitrogen trends and flow-effects of the San Joaquin river at C10, 
\item effects of wastewater treatment upgrades upstream of P8, and
\item effects of biological invasion on nutrient dynamics in Suisun Bay.
\end{inparaenum}
Each case study is built around hypotheses that results from \ac{wrtds} models are expected to support, both as a general description and for additional testing with alternative methods. 

\subsubsection{Disaggregating observed nitrogen time series}

Multiple biological and physical factors influence nutrient concentrations at different temporal scales \citep[][, references therein]{Cloern10}.  As a result, relationships between nutrients, time, and flow that could be described by quantitative methods like \ac{wrtds} are espected to be non-linear and complex.  A detailed evaluation of nitrogen dynamics at C10 using \ac{wrtds}, including the effects of flow, is expected to reveal \begin{inparaenum}[1\upshape)]
\item an annual trend independent of a seasonal trend, and 
\item varying flow contribution to nutrient dynamics, either as a difference between predicted and flow-normalized results or changes in the nutrient versus flow relationship at different annual periods.
\end{inparaenum}
Results from this case study were used to demonstrate the breadth of information that can be obtained from the observed time series with \ac{wrtds}. 

\subsubsection{Effects of wastewater treatment}

Hypothesis: Modal response of nutrient concentrations at P8 over time is result of WWTP upgrades, so we expect 1) a shift in load contributions before/after upgrade, 2) a flow-normalized annual trend at P8 to show a change concurrent with WWTP upgrades, and 3) different nitrogen species will have different changes depending on change in load outputs. See \href{http://www.waterboards.ca.gov/centralvalley/board_decisions/adopted_orders/san_joaquin/r5-2008-0086_res.pdf}{here}, but this is about Tracy
\subsubsection{Effects of biological invasions}

Hypothesis: Biological invasions by benthic filter feeders have shifted abundance and composition of phytoplankton communities in Suisun Bay, so we expect 1) decline in annual, flow-normalized chlorophyll concentrations over time coincident with increase in abundance of invaders, 2) changes in ratios of limiting nutrients (nitrogen, SiO2) suggesting different uptake rates with shift in community composition, and 3) seasonal shifts in limiting nutrients based on changes in community composition and relative abundances with seasonal succession.

\section{Results}

\subsection{Trends}

\subsection{Selected examples}

\subsubsection{Disaggregating observed nitrogen time series}

\cref{fig:dinc10}, \cref{fig:dinc10dyna}

Emphasize the information the model provides relative to the observed time series.  A distinct annual trend with a maximum in the middle of the time series is observed, with lower values at the beginning and end of the period.  The seasonal patterns generally showed that \ac{din} concentrations were highest in January with higher values at moderate to low flow rates depending on the year. Interestingly, summer and fall concentrations have showed a slight increase later in the time series (~2004-2009).  The confounding effect of flow is also very apparent such that higher flows were associated with lower concentration.  Dynaplot showed that there was always a negative assocation between the two (i.e., no modal response).  The quantile distributions showed similar trends over time in both predicted values and flow-normalized predictions, although some exceptions were observed.  In particular, high flow (1984, 2008) reduced concentrations of all quantiles but the magniutude of the effect increased at higher quantiles (i.e., the effect was disproportionate).  The opposite was observed for low flow, i.e., the ninetieth percent showed the greatest increase for low flow.     

Emphasize the summer/fall change in the 2000s, why is this?  Check \citep{Cloern07}, showed seasonal changes in early 2000s in chlorophyll (NE Pacific shifted to cool phase), is there a mechanism here with DIN? Relate to conceptual diagram.

\subsubsection{Effects of wastewater treatment}

Overall reduction in total nitrogen load was observed as a result of reduction in ammonium (\cref{fig:stock}).  Nitrate is the primary constituent of total nitrogen after 2007.  Organic nitrogen is a larger percentage of the total after nitrification. What was reduction in ammonium starting in 2002? 

Nitrogen trends at P8 shifted in response to upstream \ac{wwtp} upgrades (\cref{fig:p8trnds}), with ammonium showing the largest reduction.  Interestingly, nitrite/nitrate concentrations also showed a similar but less dramatic decrease.  Percent changes are shown in \cref{tab:p8chg}, where both nitrogen species shows large percent increases prior to \ac{wwtp} upgrades followed by decreases after upgrades with ammonium showing the largest pecentage.  Seasonally, increases prior to upgrades were most apparent in the \ac{jas} months for both analytes.  Seasonal reductions post-upgrades were also largest in \ac{jas} for nitrite/nitrate, whereas percent reductions were similar across all monthly groupings for ammonium.  

Relationships of nitrogen with flow showed the typical inverse flow/concentration dynamic with flushing at high flow, although patterns differed by nitrogen species.  Seasonal variation was more apparent for ammonium, although both typically had the highest concentrations in the winter.  Additionally, strength of the flow/nutrient relationship changed throughout the time series the year where the strongest relationship differed by analyte.  Nitrite/nitrate typically had the strongest relationship flow later in the time series, whereas ammonium had the strongest relationship with flow in the early 2000s.  

\subsubsection{Effects of biological invasions}

Data from \citep{Crauder16}, \citet{Jassby08} describes phytoplankton community changes in the upper estuary, including chlorophyll response to flow.  Figure 10 in \citet{Jassby08} showed that chlorophyll generally decreased with flow in 1980 but inreased with flow in 2000.

Note the decrease in Potamocorbula abundance in 2011, 2012.  These are wet years where abundance/biomass of the clams is driven down by lower salinity.  Contrased wtih the annual chlorophyll trends in the same years, the predicted values are above the flow-normalized trend suggesting an increase in chlorophyll with higher flow.  The potential mechanism is therefore a decrease in clam abundance with high flow that releases phytoplankon from filtration pressure.  This also explains the positive association of chlorophyll with flow in recent years (bottom right dynaplot). See suggestions in \citet{Alpine92} regarding flow/grazer relationships in the Bay. 

Further, chlorophyll trends early in the time series generally show a decrease with high flow with a distinct maximum at moderate flow.  This may suggest stratification events at moderate flow contributed to phytoplankton blooms early in the time series. Water withdrawals later in the time series could have also altered environmental conditions to reduce the frequency occurrence of stratification events.  Look into this more...

What about biomass/density relationships for Potamocorbula?  Although clam density increases throughout the period,  What about initial decrease in chlorophyll prior to clam invasion?  Is this related to water withdrawals (i.e., decrease in stratification events at moderate flow)?

\cref{fig:clmchl}, \cref{fig:d7c10trnds}, \cref{tab:d7chg}, \cref{tab:c10chg}

\section{Discussion}

Trends as percent change depend on the mean value, lower values will have larger percent changes.

Second case study showed typical inverse relationships between nutrients and flow, more flow means greater flushing and dilution of nutrient concentrations.  Conversely, low flow means less flushing and higher nutrient concentrations, although this may not always be observed if the available nutrients are biologically available.  Low-flow events during warmer months show the lowest ammonium concentrations, which corresponds to seasonal maxima in chlorophyll concentration.  A similar but weaker relationship was observed with nitrite/nitrate where increased flow was related to decreased concentration and lower concentrations overall were observed in the summer.  However, low-flow events still had higher concentrations than high-flow events in July, as compared to ammonium which was low regardless of flow.  This suggests that ammonium concentrations are driving phytoplankton production at P8.  Annual trends in chlorophyll concentration (not shown) showed an overall decrease from the 1970s to present, although a slight peak is observed in the 2000s.  This peak is likely related to the maximum ammonium concentration shown in \cref{fig:p8trnds}.  Moreover, flow/chlorophyll relationships have generally been constant throughout the period of record such that a change in flow has not been related to a change in phytoplankton production.  This suggests that nutrient loads that contribute to production at P8 are primarily from point sources at \ac{wwtp} outflows as a change in flow does not affect the load output.  But what are watershed loads?  

What do nitrogen trends mean?  Have to interpret relative to trends in other variables.  A decrease in nitrogen or constant nitrogen does not mean nitrogen inputs have stayed the same, they might actually be increasing if nitrogen.  A change in chlorohpyll relative to change in nitrogen could be informative, and even moreso, a change in silica relative to change in chlorophyll suggests diatom biomass has changed.  However, there are mismatches in these trends that suggest other processes are at play, e.g., residence times and flow inputs, etc.  Trends in Suisun relative to trends in Delta provide an example, e.g., Suisun is decrease in chlorophyll, increase in silica, increase in nitrogen, delta is decrease in silica, increase/decrease in DIN (depending on time period/season), decrease in chlorophyll, what's going on? See Senn slide 14 (from burial?). The WRTDS model lets us at least address trends in the context of season, time, and flow.  This allows for more improved interpretion relative to observing raw data. Also explain more information by looking at ammonium, nitrative/nitrite, relative to DIN. What about other variables (light level as suspended particulate matter, temperature)?

\clearpage
\begin{singlespace}
\bibliographystyle{apalike_mine}
\bibliography{refs}
\end{singlespace}
\clearpage

\begin{figure}
\centering
\includegraphics[width=1\textwidth,page=1]{figs/schematic.pdf}
\caption{Conceptual diagram illustrating use of \ac{wrtds} to decompose trends in observed nitrogen time series and potential forcing factors that can explain model output.  Results from the model are described as annual and seasonal trends, changes in flow-nutrient dynamics for different time periods, and residual variation independent of time, flow, and season. Relationships between environmental factors (climate, local, regional/historical) and nitrogen trends are more easily related to the separate components of the observed time series using results from the model. }
\label{fig:schematic}   
\end{figure}

\begin{figure}[!ht]

{\centering \includegraphics[width=0.75\textwidth]{figs/trndmap-1} 
\includegraphics[width=0.75\textwidth]{figs/trndmap-2} 

}

\caption[Percent changes in nitrogen analytes for (a) annual and (b) seasonal (monthly) periods in the record]{Percent changes in nitrogen analytes for (a) annual and (b) seasonal (monthly) periods in the record.  Changes are based on the difference between the ending and starting estimates for the flow-normalized estimates within each period.  Points are colored for direction (red increasing, green decreasing) and sized for relative magnitude. Station names are shown in the top left panel. Months for each season are Spring: MAM, Summer: JJA, Fall: SON, Winter: DJF.}\label{fig:trndmap}
\end{figure}



\begin{figure}[!ht]

{\centering \includegraphics[width=0.75\textwidth]{figs/dinc10-1} 

}

\caption{Time series of \ac{din} and flow at station C10.  Subfigure (a) shows the observed \ac{din} time series and subfigure (b) shows the annual (water year starting in October) predictions from \ac{wrtds} at different conditional quantiles ($\tau$ = 0.1, 0.5, 0.9).  The points in subfigure (b) are predictions of observed \ac{din} and the lines are flow-normalized predictions.  Subfigure (c) shows the difference between the model predictions and flow-normalized predictions at the fiftieth conditional quantile.  Subfigure (d) shows the flow time series of the San Joaquin River with a locally-estimated (loess) smooth to emphasize the long-term trend.}\label{fig:dinc10}
\end{figure}



\begin{figure}[!ht]

{\centering \includegraphics[width=\textwidth]{figs/dinc10dyna-1} 

}

\caption{Modelled relationships between \ac{din}, flow, and time at station C10.  The top figure shows the annual and seasonal variation over the entire time series and the bottom figure shows annual variation for selected months to remove seasonal variation.  Warmer colors indicate higher \ac{din} concentrations.  The y-axis on the bottom figure is truncated by the fifth and ninety-fifth percentiles of flow within each month.  Model results are for the fiftieth conditional quantile of \ac{din}.}\label{fig:dinc10dyna}
\end{figure}



\begin{figure}[!ht]

{\centering \includegraphics[width=\textwidth]{figs/stock-1} 

}

\caption[Nitrogen concentration measurements (mg L$^{-1}$) from the City of Stockton Wastewater Treatment Plant, San Joaquin County]{Nitrogen concentration measurements (mg L$^{-1}$) from the City of Stockton Wastewater Treatment Plant, San Joaquin County.  Wastewater discharge requirements were implemented in 2006 for nitrification/denitrification and tertiary filtration to convert ammonium to nitrate.}\label{fig:stock}
\end{figure}



\begin{figure}[!ht]

{\centering \includegraphics[width=\textwidth]{figs/p8trnds-1} 

}

\caption{Nitrogen trends at P8 as observed (a, top), predicted and flow-normalized estimates from \ac{wrtds} (a, bottom), and relationships with flow over time (b).  Nitrite/nitrate trends are on the left and ammonium trends are on the right.  Wastewater treatment plant upgrades at the City of Stockton (San Joaquin County), were completed in May 2007 (\cref{fig:stock}), coincident with a dramatic decrease in ammonium at P8.}\label{fig:p8trnds}
\end{figure}



\begin{figure}[!ht]

{\centering \includegraphics[width=\textwidth]{figs/clmchl-1} 

}

\caption{Trends in clam abundance and \ac{chla} concentration from 1976 to 2014 at station D7 in Suisun Bay.  Invasion by \textit{Potamocorbula amurensis} clams in the late 1980s and displacement of \textit{Corbicula fluminea} was shown by changes in clam density (a, annual means).  A coincident decrease in \ac{chla} concentration was also observed (c).  A weak but significant ($p < 0.001$) relationship between clam biomass and \ac{chla} concentration is shown in subfigure (b).  Flow relationships with \ac{chla} concentration have also changed over time (d, observations from June). Chlorophyll shows a slight positive then dominantly negative association with increasing flow (decreasing salinity) early in the time series, whereas the trend is reversed in recent years.}\label{fig:clmchl}
\end{figure}



\begin{figure}[!ht]

{\centering \includegraphics[width=\maxwidth]{figs/d7c10trnds-1} 

}

\caption{Flow-normalized trends of annual (left) and seasonal (right) variation in DIN, chl-{\it\footnotesize a}, and SiO$_2$ at station D7 (top) and C10 (bottom).  Covariation between nutrients, chl-{\it\footnotesize a}, and SiO$_2$ is observed at D7 but not C10, although an overall decrease in SiO$_2$ at C10 is shown.  Seasonal changes at D7 are most pronounced during the summer.}\label{fig:d7c10trnds}
\end{figure}



\clearpage
%%%%%%
% tables

% overall trends
%latex.default(tab[, -c(1, 2)], file = "", rowlabel = "Analyte/Station",     caption = cap.val, caption.loc = "top", rgroup = vars, n.rgroup = c(9,         9, 9), cgroup = c("Annual", "Seasonal"), n.cgroup = c(2,         4), rowname = stats, label = "tab:trnds")%
\begin{table}[!tbp]
\caption{Summaries of flow-normalized trends in nitrogen analytes for all stations and different time periods.  Summaries are averages (mg L$^{-1}$) and percent changes in parentheses (increasing in bold-italic).  Changes are based on the difference between the ending and starting estimates within each period.  See \cref{fig:trndmap} for a summary of spatial trends. Months for each season are Spring: MAM, Summer: JJA, Fall: SON, Winter: DJF.\label{tab:trnds}} 
\begin{center}
\begin{tabular}{lllcllll}
\hline\hline
\multicolumn{1}{l}{\bfseries Analyte/Station}&\multicolumn{2}{c}{\bfseries Annual}&\multicolumn{1}{c}{\bfseries }&\multicolumn{4}{c}{\bfseries Seasonal}\tabularnewline
\cline{2-3} \cline{5-8}
\multicolumn{1}{l}{}&\multicolumn{1}{c}{1976-1995}&\multicolumn{1}{c}{1996-2014}&\multicolumn{1}{c}{}&\multicolumn{1}{c}{Spring}&\multicolumn{1}{c}{Summer}&\multicolumn{1}{c}{Fall}&\multicolumn{1}{c}{Winter}\tabularnewline
\hline
{\bfseries DIN}&&&&&&&\tabularnewline
~~C10&1.3 \footnotesize{(\textit{\textbf{35}})}&1.4 \footnotesize{(-28.6)}&&1.1 \footnotesize{(-25.8)}&1.2 \footnotesize{(-6.3)}&1.5 \footnotesize{(\textit{\textbf{5.5}})}&1.7 \footnotesize{(-20.7)}\tabularnewline
~~C3&0.4 \footnotesize{(\textit{\textbf{51.1}})}&0.5 \footnotesize{(\textit{\textbf{3}})}&&0.4 \footnotesize{(\textit{\textbf{62.2}})}&0.3 \footnotesize{(\textit{\textbf{71.9}})}&0.5 \footnotesize{(\textit{\textbf{54.4}})}&0.5 \footnotesize{(\textit{\textbf{18.3}})}\tabularnewline
~~D19&0.5 \footnotesize{(-5.5)}&0.4 \footnotesize{(-30.7)}&&0.5 \footnotesize{(-33.9)}&0.3 \footnotesize{(-25.4)}&0.3 \footnotesize{(-21.7)}&0.7 \footnotesize{(-33.5)}\tabularnewline
~~D26&0.4 \footnotesize{(\textit{\textbf{0.3}})}&0.5 \footnotesize{(-19.6)}&&0.5 \footnotesize{(-19.2)}&0.3 \footnotesize{(-10.2)}&0.4 \footnotesize{(-3.5)}&0.6 \footnotesize{(-21.7)}\tabularnewline
~~D28&0.5 \footnotesize{(-3.1)}&0.4 \footnotesize{(-45.6)}&&0.5 \footnotesize{(-32.9)}&0.2 \footnotesize{(-56.5)}&0.3 \footnotesize{(-57.4)}&0.8 \footnotesize{(-40.2)}\tabularnewline
~~D4&0.4 \footnotesize{(-8.1)}&0.4 \footnotesize{(-7.5)}&&0.4 \footnotesize{(-13.6)}&0.3 \footnotesize{(\textit{\textbf{4.3}})}&0.4 \footnotesize{(-1.2)}&0.5 \footnotesize{(-15)}\tabularnewline
~~D6&0.4 \footnotesize{(\textit{\textbf{11.6}})}&0.5 \footnotesize{(-6.1)}&&0.5 \footnotesize{(-5)}&0.4 \footnotesize{(\textit{\textbf{64.1}})}&0.4 \footnotesize{(\textit{\textbf{21}})}&0.5 \footnotesize{(-14.6)}\tabularnewline
~~D7&0.4 \footnotesize{(-1.7)}&0.5 \footnotesize{(-11.8)}&&0.5 \footnotesize{(-23.8)}&0.4 \footnotesize{(\textit{\textbf{49.9}})}&0.4 \footnotesize{(\textit{\textbf{10.4}})}&0.6 \footnotesize{(-34.6)}\tabularnewline
~~P8&1.4 \footnotesize{(\textit{\textbf{81.1}})}&1.8 \footnotesize{(-15.8)}&&1.5 \footnotesize{(\textit{\textbf{35.5}})}&1 \footnotesize{(\textit{\textbf{42}})}&1.6 \footnotesize{(\textit{\textbf{55}})}&2.3 \footnotesize{(\textit{\textbf{38.7}})}\tabularnewline
\hline
{\bfseries NH$_{4}^{+}$}&&&&&&&\tabularnewline
~~C10&0.1 \footnotesize{(-37.7)}&0 \footnotesize{(-55.5)}&&0.1 \footnotesize{(-68.6)}&0 \footnotesize{(-85)}&0.1 \footnotesize{(-90.8)}&0.2 \footnotesize{(-78.8)}\tabularnewline
~~C3&0.2 \footnotesize{(\textit{\textbf{127}})}&0.3 \footnotesize{(\textit{\textbf{14.1}})}&&0.2 \footnotesize{(\textit{\textbf{165.1}})}&0.2 \footnotesize{(\textit{\textbf{133.3}})}&0.3 \footnotesize{(\textit{\textbf{107.7}})}&0.2 \footnotesize{(\textit{\textbf{81.2}})}\tabularnewline
~~D19&0.1 \footnotesize{(\textit{\textbf{35.6}})}&0.1 \footnotesize{(-32.6)}&&0.1 \footnotesize{(-32.9)}&0 \footnotesize{(-21)}&0 \footnotesize{(-6.4)}&0.1 \footnotesize{(-12)}\tabularnewline
~~D26&0.1 \footnotesize{(\textit{\textbf{58.4}})}&0.1 \footnotesize{(-18.7)}&&0.1 \footnotesize{(\textit{\textbf{3.3}})}&0.1 \footnotesize{(\textit{\textbf{38.1}})}&0.1 \footnotesize{(\textit{\textbf{21.1}})}&0.1 \footnotesize{(\textit{\textbf{0.8}})}\tabularnewline
~~D28&0.1 \footnotesize{(-10)}&0 \footnotesize{(-31.8)}&&0.1 \footnotesize{(-42.8)}&0 \footnotesize{(-36.9)}&0 \footnotesize{(-42.2)}&0.1 \footnotesize{(-35.7)}\tabularnewline
~~D4&0.1 \footnotesize{(\textit{\textbf{74.6}})}&0.1 \footnotesize{(\textit{\textbf{0.3}})}&&0.1 \footnotesize{(\textit{\textbf{54.2}})}&0 \footnotesize{(\textit{\textbf{30.8}})}&0.1 \footnotesize{(\textit{\textbf{23.8}})}&0.1 \footnotesize{(\textit{\textbf{56.9}})}\tabularnewline
~~D6&0.1 \footnotesize{(\textit{\textbf{100.2}})}&0.1 \footnotesize{(-3.5)}&&0.1 \footnotesize{(\textit{\textbf{42.2}})}&0.1 \footnotesize{(\textit{\textbf{74.1}})}&0.1 \footnotesize{(\textit{\textbf{49.2}})}&0.1 \footnotesize{(\textit{\textbf{39.8}})}\tabularnewline
~~D7&0.1 \footnotesize{(\textit{\textbf{88.2}})}&0.1 \footnotesize{(-13.3)}&&0.1 \footnotesize{(\textit{\textbf{26.8}})}&0 \footnotesize{(\textit{\textbf{25.1}})}&0.1 \footnotesize{(\textit{\textbf{4.3}})}&0.1 \footnotesize{(\textit{\textbf{16.8}})}\tabularnewline
~~P8&0.3 \footnotesize{(\textit{\textbf{267.4}})}&0.3 \footnotesize{(-85.1)}&&0.2 \footnotesize{(-51.8)}&0.1 \footnotesize{(-12)}&0.3 \footnotesize{(-44.5)}&0.6 \footnotesize{(-61.2)}\tabularnewline
\hline
{\bfseries NO$_{2}^{-}$/NO$_{3}^{2-}$}&&&&&&&\tabularnewline
~~C10&1.2 \footnotesize{(\textit{\textbf{42.8}})}&1.3 \footnotesize{(-26.9)}&&1.1 \footnotesize{(-21)}&1.2 \footnotesize{(-2.3)}&1.4 \footnotesize{(\textit{\textbf{23.3}})}&1.5 \footnotesize{(-5.6)}\tabularnewline
~~C3&0.2 \footnotesize{(\textit{\textbf{0.7}})}&0.2 \footnotesize{(-6.6)}&&0.2 \footnotesize{(\textit{\textbf{0.9}})}&0.1 \footnotesize{(\textit{\textbf{18}})}&0.1 \footnotesize{(\textit{\textbf{4.3}})}&0.2 \footnotesize{(-17.4)}\tabularnewline
~~D19&0.4 \footnotesize{(-11.7)}&0.4 \footnotesize{(-31.1)}&&0.4 \footnotesize{(-32.3)}&0.2 \footnotesize{(-36.7)}&0.3 \footnotesize{(-31.6)}&0.6 \footnotesize{(-38.4)}\tabularnewline
~~D26&0.3 \footnotesize{(-9.8)}&0.4 \footnotesize{(-20.1)}&&0.4 \footnotesize{(-23.5)}&0.2 \footnotesize{(-18.7)}&0.3 \footnotesize{(-10.5)}&0.5 \footnotesize{(-26.4)}\tabularnewline
~~D28&0.4 \footnotesize{(-10.5)}&0.4 \footnotesize{(-47.2)}&&0.5 \footnotesize{(-34.6)}&0.2 \footnotesize{(-66.3)}&0.3 \footnotesize{(-64.4)}&0.7 \footnotesize{(-45.7)}\tabularnewline
~~D4&0.3 \footnotesize{(-14.6)}&0.3 \footnotesize{(-9.8)}&&0.3 \footnotesize{(-19.2)}&0.3 \footnotesize{(\textit{\textbf{1.1}})}&0.3 \footnotesize{(-8)}&0.4 \footnotesize{(-25.9)}\tabularnewline
~~D6&0.3 \footnotesize{(-5.6)}&0.4 \footnotesize{(-8)}&&0.4 \footnotesize{(-14.5)}&0.3 \footnotesize{(\textit{\textbf{38.8}})}&0.3 \footnotesize{(\textit{\textbf{4}})}&0.4 \footnotesize{(-23.2)}\tabularnewline
~~D7&0.4 \footnotesize{(-13.4)}&0.4 \footnotesize{(-13.6)}&&0.4 \footnotesize{(-27.3)}&0.3 \footnotesize{(\textit{\textbf{28.7}})}&0.4 \footnotesize{(-23.7)}&0.5 \footnotesize{(-46.8)}\tabularnewline
~~P8&1.1 \footnotesize{(\textit{\textbf{59.8}})}&1.5 \footnotesize{(\textit{\textbf{3.3}})}&&1.2 \footnotesize{(\textit{\textbf{53.9}})}&1 \footnotesize{(\textit{\textbf{38.8}})}&1.4 \footnotesize{(\textit{\textbf{58.3}})}&1.8 \footnotesize{(\textit{\textbf{62.3}})}\tabularnewline
\hline
\end{tabular}\end{center}

\end{table}


% trends in no23, nh at p8
%latex.default(totab, file = "", rowlabel = "Period", caption = cap.val,     caption.loc = "top", rgroup = segs, n.rgroup = c(2, rep(4,         2)), cgroup = vars, n.cgroup = c(2, 2), rowname = cats,     label = "tab:p8chg")%
\begin{table}[!tbp]
\caption{Summaries of flow-normalized trends in nitrite/nitrate and ammonium (mg L$^{-1}$) concentrations before and after \ac{wwtp} upgrades upstream of station P8. Upgrades were completed in 2006 at the City of Stockton \ac{wwtp} (San Joaquin County, \cref{fig:stock}).  Summaries are means and percent changes based on annual means within the pre- and post-upgrade time periods (1976-2007, 2008-2013).  Increasing values are in bold-italics. Months for each season are Spring: MAM, Summer: JJA, Fall: SON, Winter: DJF.\label{tab:p8chg}} 
\begin{center}
\begin{tabular}{lllcll}
\hline\hline
\multicolumn{1}{l}{\bfseries Period}&\multicolumn{2}{c}{\bfseries NO$_{2}^{-}$/NO$_{3}^{2-}$}&\multicolumn{1}{c}{\bfseries }&\multicolumn{2}{c}{\bfseries NH$_{4}^{+}$}\tabularnewline
\cline{2-3} \cline{5-6}
\multicolumn{1}{l}{}&\multicolumn{1}{c}{Mean}&\multicolumn{1}{c}{\% change}&\multicolumn{1}{c}{}&\multicolumn{1}{c}{Mean}&\multicolumn{1}{c}{\% change}\tabularnewline
\hline
{\bfseries Annual}&&&&&\tabularnewline
~~1976-2007&1.3&\textit{\textbf{74.3}}&&0.34&\textit{\textbf{93.1}}\tabularnewline
~~2008-2013&1.5&-7.3&&0.11&-62\tabularnewline
\hline
{\bfseries Seasonal, pre}&&&&&\tabularnewline
~~Spring&1.24&\textit{\textbf{79.5}}&&0.25&\textit{\textbf{57.1}}\tabularnewline
~~Summer&0.99&\textit{\textbf{95}}&&0.09&\textit{\textbf{143.5}}\tabularnewline
~~Fall&1.32&\textit{\textbf{77}}&&0.29&\textit{\textbf{137.7}}\tabularnewline
~~Winter&1.67&\textit{\textbf{69.2}}&&0.69&\textit{\textbf{72.5}}\tabularnewline
\hline
{\bfseries Seasonal, post}&&&&&\tabularnewline
~~Spring&1.27&\textit{\textbf{0.9}}&&0.09&-49.2\tabularnewline
~~Summer&0.92&-18.7&&0.06&-49.2\tabularnewline
~~Fall&1.58&-8.7&&0.1&-58.2\tabularnewline
~~Winter&2.17&-5.1&&0.18&-62.1\tabularnewline
\hline
\end{tabular}\end{center}

\end{table}


% trends in din, chl, sio2 at d7
%latex.default(totab, file = "", rowlabel = "Period", caption = cap.val,     caption.loc = "top", rgroup = segs, n.rgroup = c(1, rep(4,         2)), cgroup = vars, n.cgroup = c(2, 2, 2), rowname = cats,     label = "tab:d7chg")%
\begin{table}[!tbp]
\caption{Summaries of flow-normalized trends in dissolved inorganic nitrogen (mg L$^{-1}$), chlorophyll ($\mu$g L$^{-1}$), and silicon dioxide (mg L$^{-1}$) concentrations for different time periods at station D7. Summaries are means and percent changes based on annual means within the time periods.  Increasing values are in bold-italics. Months for each season are Spring: MAM, Summer: JJA, Fall: SON, Winter: DJF.\label{tab:d7chg}} 
\begin{center}
\begin{tabular}{lllcllcll}
\hline\hline
\multicolumn{1}{l}{\bfseries Period}&\multicolumn{2}{c}{\bfseries DIN}&\multicolumn{1}{c}{\bfseries }&\multicolumn{2}{c}{\bfseries Chl-\textit{a}}&\multicolumn{1}{c}{\bfseries }&\multicolumn{2}{c}{\bfseries SiO$_2$}\tabularnewline
\cline{2-3} \cline{5-6} \cline{8-9}
\multicolumn{1}{l}{}&\multicolumn{1}{c}{Mean}&\multicolumn{1}{c}{\% change}&\multicolumn{1}{c}{}&\multicolumn{1}{c}{Mean}&\multicolumn{1}{c}{\% change}&\multicolumn{1}{c}{}&\multicolumn{1}{c}{Mean}&\multicolumn{1}{c}{\% change}\tabularnewline
\hline
{\bfseries All}&&&&&&&&\tabularnewline
~~1976-2013&0.4&-7.8&&4.1&-76.7&&12.7&\textit{\textbf{21.8}}\tabularnewline
\hline
{\bfseries Annual}&&&&&&&&\tabularnewline
~~1976-1985&0.4&-22.5&&9.3&-57.1&&10.9&-0.1\tabularnewline
~~1986-1994&0.4&\textit{\textbf{29.4}}&&3&-61.7&&12.2&\textit{\textbf{18.6}}\tabularnewline
~~1995-2003&0.5&-0.1&&2&\textit{\textbf{21.1}}&&13.7&\textit{\textbf{5.2}}\tabularnewline
~~2004-2013&0.5&-11.9&&2.4&\textit{\textbf{34.2}}&&13.7&-1.7\tabularnewline
\hline
{\bfseries Seasonal}&&&&&&&&\tabularnewline
~~Spring&0.5&-23.8&&4.2&-60.3&&14.5&\textit{\textbf{0.6}}\tabularnewline
~~Summer&0.4&\textit{\textbf{49.9}}&&6.4&-82.3&&11.2&\textit{\textbf{51.3}}\tabularnewline
~~Fall&0.4&\textit{\textbf{10.4}}&&3.7&-84.9&&11.1&\textit{\textbf{43.3}}\tabularnewline
~~Winter&0.6&-34.6&&2&-63.8&&14.2&\textit{\textbf{4.2}}\tabularnewline
\hline
\end{tabular}\end{center}

\end{table}


% trends in din, chl, sio2 at c10
%latex.default(totab, file = "", rowlabel = "Period", caption = cap.val,     caption.loc = "top", rgroup = segs, n.rgroup = c(1, rep(4,         2)), cgroup = vars, n.cgroup = c(2, 2, 2), rowname = cats,     label = "tab:c10chg")%
\begin{table}[!tbp]
\caption{Summaries of flow-normalized trends in dissolved inorganic nitrogen (mg $L^{-1}$), chlorophyll ($\mu$g $L^{-1}$), and silicon dioxide (mg $L^{-1}$) concentrations for different time periods at station C10. Summaries are means and percent changes based on annual means within the time periods.  Increasing values are in bold-italics. Months for each season are Spring: MAM, Summer: JJA, Fall: SON, Winter: DJF.\label{tab:c10chg}} 
\begin{center}
\begin{tabular}{lllcllcll}
\hline\hline
\multicolumn{1}{l}{\bfseries Period}&\multicolumn{2}{c}{\bfseries DIN}&\multicolumn{1}{c}{\bfseries }&\multicolumn{2}{c}{\bfseries Chl-\textit{a}}&\multicolumn{1}{c}{\bfseries }&\multicolumn{2}{c}{\bfseries SiO$_2$}\tabularnewline
\cline{2-3} \cline{5-6} \cline{8-9}
\multicolumn{1}{l}{}&\multicolumn{1}{c}{Mean}&\multicolumn{1}{c}{\% change}&\multicolumn{1}{c}{}&\multicolumn{1}{c}{Mean}&\multicolumn{1}{c}{\% change}&\multicolumn{1}{c}{}&\multicolumn{1}{c}{Mean}&\multicolumn{1}{c}{\% change}\tabularnewline
\hline
{\bfseries All}&&&&&&&&\tabularnewline
~~1976-2013&1.4&-10.1&&33.2&-20.6&&14.4&-19.1\tabularnewline
\hline
{\bfseries Annual}&&&&&&&&\tabularnewline
~~1976-1985&1.3&\textit{\textbf{22.8}}&&27.5&-21.6&&15.8&-3.2\tabularnewline
~~1986-1994&1.4&\textit{\textbf{1.2}}&&35.3&\textit{\textbf{31.5}}&&14.8&-4.1\tabularnewline
~~1995-2003&1.6&\textit{\textbf{16.4}}&&41&-27.5&&14.2&-3.3\tabularnewline
~~2004-2013&1.3&-36.3&&30.2&-12.8&&13.2&-7.6\tabularnewline
\hline
{\bfseries Seasonal}&&&&&&&&\tabularnewline
~~Spring&1.1&-25.8&&26.2&\textit{\textbf{3.1}}&&13.2&-20\tabularnewline
~~Summer&1.2&-6.3&&79.6&-23.2&&13.9&-25.4\tabularnewline
~~Fall&1.5&\textit{\textbf{5.5}}&&19&-39.8&&15.6&-18\tabularnewline
~~Winter&1.7&-20.7&&7.7&\textit{\textbf{7.3}}&&15.1&-14.1\tabularnewline
\hline
\end{tabular}\end{center}

\end{table}


\end{document}
