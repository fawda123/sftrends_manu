\documentclass[letterpaper,12pt]{article}
\usepackage[paperwidth=8.5in,paperheight=11in,top=1in,bottom=1in,left=1in,right=1in]{geometry}
\usepackage{setspace}
\usepackage[colorlinks=true,allcolors=Blue]{hyperref}
\usepackage{lineno}
\usepackage{cleveref}
\usepackage{acronym}
\usepackage{paralist}
\usepackage{bm}
\usepackage{fixltx2e}

%acronyms
% \acrodef{}{}

\setlength{\parskip}{5mm}
\setlength{\parindent}{0in}

\newcommand{\Bigtxt}[1]{\textbf{\textit{#1}}}

\begin{document}
\raggedright

% \title{}
% \author{}
% \maketitle

{\it Response to reviewer comments ``Four decades of water quality change in the upper San Francisco Estuary'' Beck MW, Jabusch, TW, Trowbridge PR, Senn DB.}

{\it We thank the editor and reviewer for providing helpful comments on our manuscript.  Responses to these comments are shown in italics.}

\Bigtxt{Editor's comments:}

Thank you for submitting your manuscript to Estuarine, Coastal and Shelf Science. We have completed the review of your manuscript. One reviewers recommended rejection, while the other was more lenient. If you can address the comments of Reviewer 1 adequately I will consider the MS for another round of reviews by the same reviewers. A summary is appended below. Particularly, the paper is not suitable for publication because ``.....in its present form unless there is a substantial revision to better explain/identify how this research advances our knowledge on nutrient / phytoplankton dynamics in estuarine ecosystems''.  While revising the paper please consider the reviewers' comments carefully. We look forward to receiving your detailed response and your revised manuscript.

{\it We appreciate the concern of reviewer 1 regarding significance of this work beyond the study system.  Our revision addresses this primary concern by making the novel contributions of this work beyond San Francisco estuary more clear.  This is accomplished by emphasizing the novelty of application of WRTDS to tidal systems, both in practice and interpretation of the results. We have also addressed the minor comments from both reviewers, including requested revisions to the figures.
}

\Bigtxt{Reviewer 1:}

The authors conducted a detailed data/trend analysis of nutrient variability in the upper San Francisco Estuary (SFE) based on observations collected during the last four decades. Overall, this is a very nicely written, organized and presented manuscript and I don't have major comments related to the content or analysis presented in the manuscript. However, I am very concerned that the paper seems to be a study of only local significance as it is difficult to identify new methodologies or findings of widespread impact. The method implemented for the trend analyses, namely WRTDS, has been previously used in other investigations to describe decadal trends in rivers and tidal systems (see references in MS-Lines 92-96). So, unfortunately, it doesn't seems that the implementation of WRTDS for the SFE can be considered as a novel approach. The conclusions and analyses presented from the use of WRTDS seem to be supported by the data (e.g. DIN changes at P8 following the implementation of the WWTP), but again, it is difficult to identify new causal relationships or new information that can be used to better understand nutrient dynamics in other systems around the world.  

Because of the limitations identified above, and having in mind that the journal discourages the submission of research of mainly local significance, I believe the paper should be declined in its present form unless there is a substantial revision to better explain/identify how this research advances our knowledge on nutrient / phytoplankton dynamics in estuarine ecosystems. 

{\it We greatly appreciate these comments and we have revised our paper to address these concerns.  In particular, the application of WRTDS to tidal systems is indeed a novel application that has only been reported in two publications (Beck and Hagy 2016, Beck and Murphy 2017).  Our earlier draft did not sufficiently emphasize that WRTDS has not been fully explored in tidal systems. Our application of WRTDS to the San Francisco Delta presented new challenges and novel insights.  We have added a section to the discussion to make these more clear:

``Use of WRTDS to evaluate long-term trends in water quality has not been extensive in estuarine systems.  To date, WRTDS has only been applied to evaluate eutrophication trends in Tampa Bay, Florida (Beck and Hagy 2015) and the Patuxent River Estuary (Chesapeake Bay, Beck and Murphy 2017). Unique lessons were learned from the application to these two systems, including \begin{inparaenum}[1\upshape)]  
\item use of salinity as an adequate tracer of both tidal forcing and freshwater inputs on indicators of water quality, 
\item inclusion of quantile models as an extension to evaluate differences in the lower and upper conditional distributions of water quality changes, and 
\item the ability to hypothesize pollutant sources with changes in the response of water quality to flow over time. 
\end{inparaenum}
The application of WRTDS to monitoring data from the Delta is the first to apply the method to a large temperate estuary on the Pacific coast.  

New challenges were encountered related to the geographic location and the unique characteristics of SFE that furthered our understanding of how the method can be used for trend analysis in tidal waters.  The SFE is a macrotidal system that has daily variation in water depth typical of coastal systems on the Pacific Coast (i.e., $>$ 1 m).  Previous applications of WRTDS in tidal waters were limited to systems with substantially smaller tidal amplitudes (i.e., $<$ 1 m), demonstrating that WRTDS can be applied in systems with a wide range of tidal influences.  This point is particularly relevant given that hydrologic factors are often key determinants of water quality changes in coastal systems, suggesting the method is robust in the presence of large tidal swings. The SFE is also unique in that excess turbidity, rather than nutrients, is the dominant limiting factor of primary production.  This differs substantially from Tampa Bay and Chesapeake Bay, neither of which are light-limited from sedimentation.  Our case study in Suisun Bay provided a unique demonstration of WRTDS to describe chlorophyll dynamics as a function of multiple factors unrelated to nutrients by evaluating the effects of biological invasion on water quality and the complex interaction of flow on both. This result provided the first application of WRTDS in tidal waters to evaluate multiple factors that are included explicitly in the model (e.g, flow) and those that are not (e.g., biological forcing). As such, WRTDS has the potential to inform the understanding of nutrient and phytoplankton dynamics in coastal systems that vary greatly in physical, chemical, and biological characteristics.''

We also want to emphasize that the previous draft did indicate an important novel contribution of this work beyond the San Francisco Estuary.  In particular, our result that showed trend analyses conducted on observed data could lead to different conclusions relative to flow-normalized data is particularly important and applies independent of location.  This was described previously on lines 10-11 in the abstract, lines 319-337 in the results, and lines 431-439, 509-511 in the discussion.  
}

Minor comments

Check figure 1. Seems disorganized. Map of California in wrong place and not at scale.

{\it Figure 1 was simplified and the California map was given its own inset.}

Line  270 - 271. The figures don't seem to support this statement. If the colors are correct, then I see NH4 as the predominant form of Nitrogen in most stations. 

{\it We have changed the color scheme to make this distinction more clear.  This was also noted by reviewer 2.  The dominance of nitrite/nitrate (grey) relative to ammonium (dark blue) is now apparent.}


\Bigtxt{Reviewer 2:}

This is a very well written article on an important subject.  It makes a significant contribution to the literature with regard to analysis of nutrient loading to an estuary.  This is a very well studied estuary already, and the results of the study are not at all unexpected, but it is an excellent application of new analysis methodology for estuaries subject to temporal, seasonal, and hydrologically influenced changes.  There is not much that I could find in the way of suggested improvements.  I would recommend it for publication with minor revisions as follows.

p. 7, line 130   Seasonally, inflow from the watershed ...

This a long sentence with three clauses that don't quite follow from one to the next, in my opinion.  I think the manuscript would read better by splitting the sentence into two.

{\it Changed to ``Seasonally, inflows from the watershed peak in the spring and early summer from snowmelt.  Long-term trends have shown that consumption, withdrawals, and export have steadily increased from 1960 to present, although climate effects have contributed to inter-annual variability (Cloern and Jassby 2012).''}

p. 10, line 186 - I think readers would appreciate another sentence or two on how the weighting within the time window is accomplished.  I assume it is with some sort of exponentially decaying time function, but there isn't any description provided.  A short explanation is all that's needed.  

{\it Details were added: ``Weights within each window are estimated using a tri-cube weighting function, which is similar to a bell curve except the tail ends do not asymptote at zero (Hirsch et al. 2010).  This ensures that observations outside of the window are given zero weight, as compared to an infinitely small value.  The final vector of weights used for each regression is the product of the three separate weight vectors for time, season, and flow.''}

p.10, line 195 - This is a follow-up to the previous comment.  Adding a mention of the time constant in the description of the method will help setup the presentation of the times used for averaging.  I think you should add some information here on how these time windows were chosen.

{\it An additional sentence was added to the previous paragraph to describe how time and season values are evaluated: ``Annual values are described using decimal time as a continuous numeric value (e.g., July 1st, 2017 is 2017.5) and season is described as a numeric value for day of year.''. 

The text on line 195 was also modified for clarity: `` Model predictions were evaluated as monthly values that were the same as the day of sampling in the observed data and as annual values that averaged monthly predictions within each water year (October to September).''
}

Figure 2.  More contrast in colors (perhaps blue and gray) would make it easier to distinguish ammonia and nitrate in the time history.  As is, it is somewhat difficult to see

{\it Figure was modified, also in responese to reviewer 1.}

p.18, line  In the text, I believe in the introduction, you mention one invading asian clam (potamocorbula), but the second clam genera (corbicula) isn't mentioned until the results and in figure and figure caption.  Is corbicula a native clam that was displaced?  It is not clear from the text what we are to make of the decrease in abundance of corbicula.  A few explanatory sentences in the introduction and/or the results sections are needed.

{\it Both are non-native invaders with different competitive advantages.  This was clarified in the text: ``Invasion in the 1980s showed a clear increase of \textit{P. amurenusis} at D7 that coincided with a reduction in abundance of the non-native and previously established Asian Clam (\textit{Corbicula fluminea}) (Fig. 7a).''}

\end{document}